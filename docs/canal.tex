% pdflatex canal.tex
\documentclass[a4paper]{book}
\usepackage[utf8]{inputenc}
\usepackage[english]{babel}
\usepackage{a4wide}
\usepackage{makeidx}
\usepackage{hyperref}
\usepackage{graphicx}
\usepackage{import}
\usepackage{amsmath}
\usepackage{amssymb}
\usepackage{txfonts}
\usepackage{float}
\usepackage{doxygen-lib/doxygen}

\fancyfoot[RE]{\fancyplain{}{}}
\fancyfoot[LO]{\fancyplain{}{}}

\setcounter{tocdepth}{2}

\makeindex

\title{Canal\\
\vskip 1em
\large{a static code analysis tool}}
\author{Karel Klíč}

\begin{document}
\maketitle

\tableofcontents

\cleardoublepage

\chapter{Overview}

For a sufficiently complex software system, its maintainability and
extensibility is limited by our ability to understand and correctly
approximate the behaviour of the system, trace the impact of system
parts to each other, control the impact of modifications, ensure
correctness of the critical parts, and fixing bugs before they cause
serious consequences in production.

The maintainability and extensibility is affected by the programming
language of implementation.  Efficient low-level languages such as C
and C++ increase the complexity of the system by being closely aligned
with hardware. Systems must handle memory management, operate with
machine-dependent integers and floating point numbers, and use system
calls with complex invariants and interdependencies.

Canal is a framework combining existing static analysis techniques
improving maintainability, understanding, traceability and correctness
in a coherent manner.  The purpose of the framework is to make
existing techniques accessible and evaluable, to support the
implementation of new techniques, and to allow experiments.
Currently, techniques are often presented without proper experiments
on real-world complex systems, or just with a proprietary
implementation that cannot be investigated.  The consequence is that
actual applicability of many techniques for industrial use is unknown.
We hope to change this situation.

\section{Use cases}
\subsection{Analysis of program behaviour}
You can hook on the fixpoint of function calls to inspect the
calculated abstract values.  You can get abstract values of function
call parameters.

\subsection{Comparison with a specification}
A set of pre- and post-conditions for functions, and variable-based or
module-based automata.  This can be defined for certain function or
library, and library/function users are watched to conform to the
specification.

\subsection{Conformance to environment constraints}
Double free, memory leaks, buffer overflow and underflow, division by
zero, invalid access to memory, locking and concurrency errors,
uncaught exceptions.

\section{Artifacts}
\subsection{Source code}
\subsection{Specification}
\subsection{Models}

\chapter{Installation}

%\section{Installation from source code on Microsoft Windows XP}

%\section{Installation from source code on Mac OS X}

%\section{Installation from source code on Ubuntu 12.04}

%Canal can be built and installed on Ubuntu 12.04.

\section{Installation from source code on Red Hat Enterprise Linux 6}

Canal can be built, installed, and developed on a computer with the
Red Hat Enterprise Linux 6 operating system.

\paragraph*{Prerequisites}
Specific software packages are required by the build process and
should be installed prior to building Canal:
\begin{itemize}
\item The core library requires \texttt{llvm-devel} and \texttt{clang}
  packages, which can be obtained from Extra Packages for Enterprise
  Linux (or EPEL) software repository.
\item The command-line user interface tool requires
  \texttt{elfutils-devel} and \texttt{readline-devel} packages.
\item The documentation requires \texttt{doxygen}, \texttt{graphviz},
  and \texttt{texlive-latex} packages.
\end{itemize}
The compiled and installed Canal requires \texttt{llvm},
\texttt{clang}, \texttt{elfutils}, and \texttt{readline} packages.

\section{Installation from source code on Fedora 17}

Canal can be built, installed, and developed on a computer with the
Fedora 17 operating system.

\paragraph*{Prerequisites}
Specific software packages are required by the build process and
should be installed prior to building Canal:
\begin{itemize}
\item The core library requires \texttt{llvm-devel} and \texttt{clang}
  packages.
\item The command-line user interface tool requires
  \texttt{elfutils-devel} and \texttt{readline-devel} packages.
\item The documentation requires \texttt{doxygen}, \texttt{graphviz},
  and \texttt{texlive-latex} packages.
\end{itemize}
The compiled and installed Canal requires \texttt{llvm},
\texttt{clang}, \texttt{elfutils}, and \texttt{readline} packages.


\part{Concepts}

\chapter{Preliminaries}

Definitions from the order theory.  More details can be found in
\cite{DP02}.

A binary relation $\sqsubseteq$ is \emph{reflexive} on a set
$\mathcal{D}$ if every element is related to itself: $a \sqsubseteq a$
for all $a \in \mathcal{D}$.  A binary relation $\sqsubseteq$ is
\emph{antisymmetric} on a set $\mathcal{D}$ if the following
implication holds: $a \sqsubseteq b$ and $b \sqsubseteq a$ implies $a
= b$.  A binary relation $\sqsubseteq$ is \emph{transitive} on a set
$\mathcal{D}$ if whenever an element $a$ is related to an element $b$,
and $b$ is in turn related to an element $c$, then $a$ is also related
to $c$: $a \sqsubseteq b$ and $b \sqsubseteq c$ implies $a \sqsubseteq
c$.

A \emph{partial order} $\sqsubseteq$ is a binary relation on a set
$\mathcal{D}$ which is reflexive, antisymmetric and transitive.  A
\emph{partial ordered set} or \emph{poset} for short is an ordered
pair $(\mathcal{D}, \sqsubseteq)$ of a set $\mathcal{D}$ together with
a partial ordering $\sqsubseteq$.

An element $a$ in a poset $(\mathcal{D}, \sqsubseteq)$ is called
\emph{maximal} if it is not less than any other element in
$\mathcal{D}$: $\nexists b \in \mathcal{D}, a \sqsubset b$.  If there
is an unique maximal element, we call it the \emph{greatest element}
and denote it by $\top$.  Similarly, an element $a$ in a poset
$(\mathcal{D}, \sqsubseteq)$ is called \emph{minimal} if it is not
greater than any other element in $\mathcal{D}$: $\nexists b \in
\mathcal{D}, b \sqsubset a$.  If there is an unique minimal element,
we call it the \emph{least element} and denote it by $\bot$.

Let $(\mathcal{D}, \sqsubseteq)$ be a poset and $A \subseteq
\mathcal{D}$.  An element $u \in \mathcal{D}$ is an \emph{upper bound}
of $A$ if $a \sqsubseteq u$ for all elements $a \in A$. The
\emph{least upper bound} or \emph{lub} for short is an element $x$
that is an upper bound on a subset $A$ and is less than all other
upper bounds on $A$; such an element is denoted by $\bigsqcup A$.
Similarly, an element $l \in \mathcal{D}$ is a \emph{lower bound} of
$A$ if $l \sqsubseteq a$ for all elements $a \in A$. The
\emph{greatest lower bound} or \emph{glb} for short is an element $x$
that is a lower bound on a subset $A$ and is greater than all other
lower bounds on $A$; such an element is denoted by $\bigsqcap A$.

A \emph{lattice} $(\mathcal{D}, \sqsubseteq, \sqcup, \sqcap)$ is a
partially ordered set in which any two elements $a, b \in \mathcal{D}$
have both a least upper bound, denoted by $a \sqcup b$, and a greatest
lower bound, denoted by $a \sqcap b$.  A \emph{complete lattice}
$(\mathcal{D}, \sqsubseteq, \sqcup, \sqcap, \bot, \top)$ is a
partially ordered set in which every subset $A \subseteq \mathcal{D}$
has a least upper bound and a greatest lower bound.

A \emph{fixpoint} of a function $F$ is an element $X$ such that $F(X)
= X$.

A function $F \in \mathcal{D}_1 \to \mathcal{D}_2$ between two posets
$(\mathcal{D}_1, \sqsubseteq_1)$ and $(\mathcal{D}_2, \sqsubseteq_2)$
is \emph{monotonic} if $X \sqsubseteq_1 Y \implies F(X) \sqsubseteq_2
F(Y)$.  A function $F \in \mathcal{D}_1 \to \mathcal{D}_2$ is
\emph{strict} if $F(\bot_1) = \bot_2$.  A function $F \in
\mathcal{D}_1 \to \mathcal{D}_2$ is \emph{continuous} if it preserves
the existing limits of increasing chains $(X_i)_{i \in I}$:
$F(\bigsqcup_1 \{ X_i \mid i \in I \}) = \bigsqcup_2 \{ F(X_i) \mid i
\in I \}$ whenever $\bigsqcup_1 \{ X_i \mid i \in I \}$ exists.


\chapter{LLVM}
LLVM was first presented in \cite{LA04}.

Formalized in \cite{ZNMZ12}.

\chapter{Abstract Interpretation}
Our abstract interpreter comes in four flavours:
\begin{description}
\item[Context-insensitive flow-insensitive] For every function in a
  program, the fixpoint is calculated with a single set of abstract
  values that encompasses all function calls.
\item[Context-insensitive flow-sensitive] For every function in a
  program, the fixpoint is calculated with a single set of abstract
  values that encompasses all function calls, but every possible path
  through the function is calculated separately.
\item[Context-sensitive flow-insensitive] The fixpoint is calculated
  with a set of abstract values specifically created for every
  function call.
\item[Context-sensitive flow-sensitive] The fixpoint is calculated
  with a set of abstract values specifically created for every
  possible path in a function call.  Path conditions are taken into
  account.
\end{description}

Abstract interpreter can be either operational or equation-based.  Our
interpreter is operational.

\subsection{Tuning}
The precision of abstract interpreter is greately tunable. Here are
the aspects to consider:
\begin{description}
\item[Interpreter flavour] Context-sensitivity and flow-sensitivity
  increase both precision and complexity.
\item[Widening and narrowing] Selection and parameters of widening and
  narrowing operators affect both precision and complexity.
\item[Relations in abstract domains] Type and number of relations in
  abstract domains affect both precision and complexity.
\item[Memory for abstract domains] Parameters of some abstract domains
  allow to trade memory for better precision.
\end{description}

Maximal precision of abstract interpreter is same as for symbolic
executor, but abstract interpreter is more tunable.

\subsection{Context sensitivity}
Context sensitivity is achieved by keeping a function call stack.
Every stack frame keeps the complete state of a function fixpoint
calculation (all local and global variables).  When a function call is
reached during the fixpoint computation and function call parameters
are already initialized, a new frame is placed on the top of stack and
the called function is interpreted with the provided parameters.

\subsection{Flow sensitivity}


\chapter{Abstractions}

\section{Multi threading}
Multi-threading abstraction for Abstract Interpretation appeared in
\cite{M11}.

\section{Memory}
Memory abstraction appeared in \cite{M06}.

Our memory abstraction for abstract interpretation recognizes four
kinds of memory:
\begin{description}
\item[Register-like stack memory] This is function-level memory that
  is released automatically when function returns.  We denote such a
  memory by LLVM-style names starting with the percent sign
  \texttt{\%}.  Memory either has a name (e.g. \texttt{\%result}) or a
  number is generated to serve as a name (e.g. \texttt{\%32} denotes
  thirty-second unnamed instruction call in a function).
\item[Stack memory allocated by \texttt{alloca}] This is also a
  function-level memory that is released automatically when function
  returns.  The difference to register-like stack memory is that this
  memory is accessed by LLVM exclusively via pointers.  We denote such
  a memory by names starting with \texttt{\%\^}.  Every piece of
  memory has a name corresponding to the place where the memory has
  been allocated (\texttt{alloca} has been called).  So if the memory
  has been allocated by an instruction call \texttt{\%ptr = alloca
    i32, align 4}, it can be denoted by \texttt{\%\^{}ptr}.
\item[Global variables] Global variables are module-wise and are valid
  for the whole program run.  We denote such a memory by LLVM-style
  names starting with \texttt{@}.
\item[Heap memory] Heap memory is also valid for the whole program
  run.  We denote such a memory by names starting by \texttt{@\^}.
  Every piece of memory has a name corresponding to the place where
  the memory has been allocated (\texttt{malloc} or similar function
  has been called).  Name of the function is also included in the
  place name, so if a function \texttt{createString} contains an
  instruction call \texttt{\%result = call i8* @malloc(i32 1)}, we can
  denote the memory allocated on this place by
  \texttt{@\^{}createString:result}.
\end{description}

As it can be seen from the style of memory denotation, every piece of
memory is associated with a place in the program.  This means all
operations affecting a memory block allocated at certain place forms a
single abstract value.  Context-sensite abstract interpretation helps
to increase the precision of this memory abstraction.

\section{Arrays}

\section{Structures}

\section{Integers}
Precise machine integer abstraction appeared in \cite{M12}.

\section{Floating-point numbers}
Precise machine floating-point abstraction appeared in \cite{M12}.

\section{Pointers}
Pointer can be casted to a number via the \texttt{ptrtoint}
instruction.  Usually, the resulting memory offset is used to achieve
pointer arithmetics that are not available via \texttt{getelementptr}
semantics.

\chapter{Wishlist}
\begin{description}
\item[Lazy model-checking abstract value] Allow to investigate just a
  single function, taking into account all possible parameter values
  and shapes (perhaps limited by a pre-condition).  Parameter values
  and shapes must be smartly provided depending on the boundary
  requirements of the checked code.  This allows a kind of model
  checking, and use of model checking algorithms and ideas.
\item[Widening operators] Implement widening operators for integers
  and other abstract domains as required.
\item[Narrowing operators] Implement narrowing operators for integers
  and other abstract domains as required.
\item[String abstractions] Implement abstract domains specific for C
  strings.
\item[Weakly relational numeric abstractions] Implement weakly
  relational integer and floating-point abstract domains.
\item[Basic block abstraction] Implement basic block summaries that
  speed-up the static analysis.
\item[Function abstraction] Implement function summaries that speed-up
  the static analysis.
\item[Parallelization] Make abstract interpreter to use multiple
  threads for fixpoint calculation on the right level.
\item[Concurrency check] Add support for checking of multi-threaded
  programs.
\end{description}

\part{Implementation}

\chapter{Overview}

Canal can be used for a static analysis of real-world complex software
systems written in efficient low-level languages C and C++.  It uses
the LLVM intermediate representation for the static analysis.

Canal is implemented in the C++ language as defined in the C++98
standard (ISO/IEC 14882:1998).  It uses the C++ standard library and
some additional libraries:
\begin{itemize}
\item LLVM core libraries.  All versions from 2.8 up to 3.1 are
  supported.
\item Clang compiler.  Any version working with a supported version of
  LLVM should work.
\item GNU readline.  Any BSD-licensed reimplementation can be used as
  an alternative.
\item elfutils.  This library is used only on Linux-based operating
  systems.
\end{itemize}

\import{doxygen-lib/}{refman}
\import{doxygen-tool/}{refman}

\chapter{Known Bugs}
Pointers should have the possibility to be set to top.


\cleardoublepage
\addcontentsline{toc}{chapter}{Bibliography}
\begin{thebibliography}{9}

\bibitem{CC77} Patrick Cousot and Radhia Cousot.  Abstract
  Interpretation: A Unified Lattice Model for Static Analysis of
  Programs by Construction or Approximation of Fixpoints.  In
  \emph{POPL '77: Proceedings of the 4th ACM SIGACT-SIGPLAN symposium
    on Principles of programming languages}, 1977.

\bibitem{CC79} Patrick Cousot and Radhia Cousot.  Systematic Design of
  Program Analysis Frameworks.  In \emph{POPL '79: Proceedings of the
    6th ACM SIGACT-SIGPLAN symposium on Principles of Programming
    Languages}, 1979.

\bibitem{HCXE02} Seth Hallem, Benjamin Chelf, Yichen Xie, and Dawson
  Engler.  A System and Language for Building System-Specific, Static
  Analyses.  In \emph{PLDI '02: Proceedings of the ACM SIGPLAN 2002
    Conference on Programming language design and implementation},
  2002.

\bibitem{DP02} Brian Albert Davey and Hilary Ann
  Priestley. Introduction to Lattices and Order. 2nd ed. Cambridge
  University Press, 2002.

\bibitem{LA04} Chris Lattner and Vikram Adve.  LLVM: A Compilation
  Framework for Lifelong Program Analysis \& Transformation. In
  \emph{CGO '04: Proceedings of the International Symposium on Code
    Generation and Optimization: Feedback-directed and Runtime
    Optimization}, 2004.

\bibitem{M06} Antoine Miné.  Field-Sensitive Value Analysis of
  Embedded C Programs with Union Types and Pointer Arithmetics.  In
  \emph{LCTES '06: Proceedings of the 2006 ACM SIGPLAN/SIGBED
    conference on Language, compilers, and tool support for embedded
    systems}, 2006.

\bibitem{M11} Antoine Miné.  Static Analysis of Run-time Errors in
  Embedded Critical Parallel C Programs.  In \emph{ESOP '11:
    Proceedings of The 20th European Symposium on Programming}, 2011.

\bibitem{M12} Antoine Miné.  Abstract Domains for Bit-Level Machine
  Integer and Floating-point Operations.  In \emph{WING '12:
    Proceedings of The 4th International Workshop on Invariant
    Generation}, 2012.

\bibitem{ZNMZ12} Jianzhou Zhao, Santosh Nagarakatte, Milo
  M. K. Martin, and Steve Zdancewic.  Formalizing the LLVM
  Intermediate Representation for Verified Program Transformations. In
  \emph{POPL '12: Proceedings of the 39th annual ACM SIGPLAN-SIGACT
    symposium on Principles of programming languages}, 2012.

\end{thebibliography}

\clearpage
\addcontentsline{toc}{chapter}{Index}
\printindex

\end{document}
